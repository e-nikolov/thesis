The field of Secure Multiparty Computation provides methods for jointly
computing functions without revealing their private inputs from multple
parties. This master thesis assignment focuses on the MPyC framework for
MPC and explores various approaches for connecting the parties via the
internet. A technical survey was performed in the preparation phase to
identify viable techniques and tools to achieve that. Furthermore a test
environment dubbed \(E^3\) was developed to support the exploration
process that will take place during the implementation phase of the
assignment. It is composed of a combination of physical and virtual
machines that are able to execute a multiparty computation together
using MPyC. It employs several declarative Infrastructure as Code tools
to automate the deployment process and make it reproducible.
Specifically, Terraform is used for provisioning NixOS virtual machines
on the DigitalOcean cloud provider and Colmena is used for remotely
deploying software to them. The reference implementation described in
this report uses the Tailscale mesh VPN for connectivity, and a number
of additional implementations are planned for the next phase of the
project.
