\hypertarget{brainstorm-for-custom-solutions}{%
\chapter{Brainstorm for custom
solutions}\label{brainstorm-for-custom-solutions}}

\begin{itemize}
\tightlist
\item
  Initial state

  \begin{itemize}
  \tightlist
  \item
    Here's my identity, here are the identities of the other parties
  \end{itemize}
\item
  Desired result

  \begin{itemize}
  \tightlist
  \item
    Executed MPC
  \end{itemize}
\end{itemize}

With tailscale we'd need to

\begin{itemize}
\tightlist
\item
  Each party:

  \begin{itemize}
  \tightlist
  \item
    registers a Tailscale account
  \item
    Downloads and installs tailscale on the machine they want to run the
    MPC on
  \item
    Runs tailscale on their machine and logs into their account in order
    to link it to their own Tailnet
  \item
    Shares their Tailscale machine with the Tailnets of each of the
    other parties
  \item
    Downloads the demo they want to run
  \item
    Form the flags for running the chosen demo

    \begin{itemize}
    \tightlist
    \item
      add -P \$HOST:\$PORT for each party using their Tailscale
      hostname/virtual IP
    \end{itemize}
  \item
    Run the demo
  \end{itemize}
\end{itemize}

\begin{center}\rule{0.5\linewidth}{0.5pt}\end{center}

\begin{itemize}
\tightlist
\item
  we don't need to use the same route for the communication Party A →
  Party B and Party B → Party A
\item
  we can have something like an asynchronous STUN
\item
  Party A sends a QR code/public URL/json object/DID document containing

  \begin{itemize}
  \tightlist
  \item
    Party A's public key
  \item
    Party A's Mediator URL
  \end{itemize}
\item
  The mediator is a STUN/TURN/DERP server

  \begin{itemize}
  \tightlist
  \item
    other parties can either use it as a STUN server to find out how to
    access the hole in party A's NAT punched by the mediator
  \item
    or use it as a relayer so that Party B can send encrypted packets to
    party A via the mediator
  \end{itemize}
\item
  Instead of a QR code, the information could be stored on a public
  ledger and could be resolved via DIDs
\end{itemize}

\begin{center}\rule{0.5\linewidth}{0.5pt}\end{center}

\begin{itemize}
\tightlist
\item
  There is a generic MPC wrapper program that deals with supporting
  tasks like generating identities and pulling MPC demos
\item
  One party creates a ``lobby'' for an MPC session in their program and
  get a session id/public URL/QR code that can be shared with the other
  parties
\item
\end{itemize}

\begin{center}\rule{0.5\linewidth}{0.5pt}\end{center}

\begin{itemize}
\tightlist
\item
  Public Website is visited by everyone
\item
  They prove their identities using a SSI wallet or OIDC
\item
  They get a QR code that serves as an invitation

  \begin{itemize}
  \tightlist
  \item
    contains their STUN based endpoint and identity
  \end{itemize}
\item
  Somehow everyone needs to scan each other's qr codes
\end{itemize}

\begin{center}\rule{0.5\linewidth}{0.5pt}\end{center}

\begin{center}\rule{0.5\linewidth}{0.5pt}\end{center}

\begin{center}\rule{0.5\linewidth}{0.5pt}\end{center}

\begin{itemize}
\tightlist
\item
  One party creates a ``lobby'' for an MPC session by visiting a public
  website

  \begin{itemize}
  \tightlist
  \item
    They provide their identity via OIDC/SSI wallet
  \end{itemize}
\item
  They get a public link/QR code that can be shared with the other
  parties
\item
  The parties visit the URL and also provide their identities
\item
  The parties obtain the MPC program they want to run

  \begin{itemize}
  \tightlist
  \item
    MPC program distribution could be done separately via cloning the
    github repo?
  \item
    They could choose a DEMO and download it from the website?
  \item
    There could also be a program running on the host machines that
    deals with the source code distribution. Similar to downloading
    custom maps for warcraft 3 or dota 2?
  \item
    They could specify the source code when creating the MPC session in
    the website?
  \item
    If the demo is not symmetrical where different parties have
    different roles and need to execute different programs, the roles
    could be assigned by the host or the people could choose their
    preferred role themselves?
  \end{itemize}
\item
  The parties download a configuration file that contains information on
  how to connect to the other parties
\item
  They run the demos with the downloaded config file
\item
  A temporary Wireguard mesh VPN is created between all parties
\end{itemize}
