\hypertarget{intro}{%
\chapter{Intro}\label{intro}}

\begin{itemize}
\tightlist
\item
  Emphasize on the difference between the Solutions Test Framework (E3)
  and the actual Solutions
\item
  The Test Framework is an engineering problem of its own which deals
  with the scaffolding necessary to demonstrate the candidate solutions
  in realistic scenarios
\end{itemize}

asdf

\hypertarget{landscape-overview}{%
\chapter{Landscape Overview}\label{landscape-overview}}

{[}{[}0201-webrtc{]}{]}

\hypertarget{webrtc}{%
\section{WebRTC}\label{webrtc}}

\begin{itemize}
\tightlist
\item
  Peer to peer communications for browsers

  \begin{itemize}
  \tightlist
  \item
    can also work without a browser
  \item
    Mainly used for multimedia communications - Peer-to-peer
    Audio/Video/VoIP
  \end{itemize}
\item
  Spec -
  \href{https://www.w3.org/TR/webrtc/\#persistent-information-exposed-by-webrtc}{https://www.w3.org/TR/webrtc/}
\item
  Uses \href{notes/0202-nat}{nat} STUN/TURN/ICE
\item
  Data is encrypted
\item
  Identity

  \begin{itemize}
  \tightlist
  \item
    Session Description Protocol (SDP)
  \item
    peer certificates are generated and announced over SDP
  \item
    ICE Candidates are negotiated for STUN/TURN connections
  \end{itemize}
\item
  Not a VPN

  \begin{itemize}
  \tightlist
  \item
    I think it can't serve as a TCP/IP network overlay that other
    applications can use
  \end{itemize}
\item
  Does not require additional plugins or native apps
\item
  There seem to be many publicly available services that can be used as
  ICE servers for WebRTC

  \begin{itemize}
  \tightlist
  \item
    \href{http://stun.l.google.com:19302/}{stun.l.google.com:19302}
  \item
    \url{https://gist.github.com/zziuni/3741933}
  \end{itemize}
\item
  Examples

  \begin{itemize}
  \tightlist
  \item
    \url{https://github.com/pion/webrtc}

    \begin{itemize}
    \tightlist
    \item
      Go
    \end{itemize}
  \item
    \url{https://github.com/pojntfx/weron}
  \end{itemize}
\item
  Resources

  \begin{itemize}
  \tightlist
  \item
    \url{https://webrtcforthecurious.com/}
  \item
    \url{https://temasys.io/guides/developers/webrtc-ice-sorcery/}
  \item
    \href{https://web.dev/webrtc-basics/\#toc-rtcpeerconnection}{https://web.dev/webrtc-basics/}
  \end{itemize}
\end{itemize}

\hypertarget{network-address-translation-nat-traversal}{%
\section{Network Address Translation (NAT)
Traversal}\label{network-address-translation-nat-traversal}}

\url{https://bford.info/pub/net/p2pnat/}

\url{https://www.jordanwhited.com/posts/wireguard-endpoint-discovery-nat-traversal/}

\hypertarget{section}{%
\subsection{123 456}\label{section}}

\hypertarget{session-traversal-utilities-for-nat-stun}{%
\subsection{Session Traversal Utilities for NAT
(STUN)}\label{session-traversal-utilities-for-nat-stun}}

\begin{itemize}
\tightlist
\item
  Uses a STUN server for discovery and UDP hole-punching
\item
  Communications are peer-to-peer
\item
  Examples:

  \begin{itemize}
  \tightlist
  \item
    \url{https://github.com/shawwwn/Gole}
  \item
    \url{https://github.com/malcolmseyd/natpunch-go} \#\#\# Traversal
    Using Relays around NAT (TURN)
  \end{itemize}
\item
  Peers use a relay server as a mediator to route traffic \#\#\#
  Interactive Connectivity Establishment (ICE)
\item
  Umbrella term covering STUN/TURN and other related techniques
\end{itemize}

\hypertarget{designated-encrypted-relay-for-packets-derp}{%
\subsection{Designated Encrypted Relay for Packets
(DERP)}\label{designated-encrypted-relay-for-packets-derp}}

\begin{itemize}
\tightlist
\item
  TURN-like protocol by Tailscale
\item
  Relaying encrypted Wireguard traffic over HTTP
\item
  Routing based on the Peer's public key
\item
  Overview - \url{https://tailscale.com/kb/1232/derp-servers/}
\item
  Source code

  \begin{itemize}
  \tightlist
  \item
    \url{https://github.com/tailscale/tailscale/tree/main/cmd/derper}
  \item
    \url{https://github.com/tailscale/tailscale/blob/main/derp/derp.go}
  \end{itemize}
\end{itemize}

\hypertarget{wireguard}{%
\section{Wireguard}\label{wireguard}}

\begin{itemize}
\tightlist
\item
  VPN Protocol
\item
  Used by Tailscale
\item
  Whitepaper - \url{https://www.wireguard.com/papers/wireguard.pdf}
\item
  Built with the \href{notes/0204-noise}{Noise Protocol Framework}
\item
  Typically used as a building block in more complicated systems
\item
  Simple configuration

  \begin{itemize}
  \item
    Each peer has a public/private key pair for authentication and
    traffic encryption
  \item
    Each peer has a config file:

\begin{Shaded}
\begin{Highlighting}[]
\KeywordTok{[}\DataTypeTok{Interface}\KeywordTok{]}
\DataTypeTok{Address} \OperatorTok{=} \FloatTok{101.0}\KeywordTok{.}\DataTypeTok{0}\KeywordTok{.}\DataTypeTok{1}\ErrorTok{/}\DataTypeTok{32}
\DataTypeTok{ListenPort} \OperatorTok{=} \DecValTok{53063}
\DataTypeTok{PrivateKey} \OperatorTok{=} \DataTypeTok{ePTiXXhHjvAHdWUr8Bimk30n0gh3m241RAzsN0JZDW0}\OperatorTok{=}

\KeywordTok{[}\DataTypeTok{Peer}\KeywordTok{]}
\DataTypeTok{PublicKey} \OperatorTok{=} \DataTypeTok{BSn0ejd1Y3bKuD}\ErrorTok{+}\DataTypeTok{Xpg0ZZeOf}\ErrorTok{+}\DataTypeTok{Ies}\ErrorTok{/}\DataTypeTok{oql0NZxw}\ErrorTok{+}\DataTypeTok{SOmkc}\OperatorTok{=}
\DataTypeTok{AllowedIPs} \OperatorTok{=} \FloatTok{101.0}\KeywordTok{.}\DataTypeTok{0}\KeywordTok{.}\DataTypeTok{2}\ErrorTok{/}\DataTypeTok{32}
\DataTypeTok{Endpoint} \OperatorTok{=} \FloatTok{142.93}\KeywordTok{.}\DataTypeTok{135}\KeywordTok{.}\DataTypeTok{154}\ErrorTok{:}\DataTypeTok{38133}
\DataTypeTok{PersistentKeepalive} \OperatorTok{=} \DecValTok{25}
\end{Highlighting}
\end{Shaded}
  \end{itemize}
\item
  Creates a virtual network interface in the operating system that looks
  like an additional network card and can be used for TCP/IP
  communications
\item
  Handles the encryption of traffic

  \begin{itemize}
  \tightlist
  \item
    deals with handshakes and generating symmetric session keys
  \end{itemize}
\item
  Cryptokey routing - associates public/private key pairs with IP
  addresses
\item
  Out of scope:

  \begin{itemize}
  \tightlist
  \item
    key distribution - managed manually or via other software that
    builds on top of wireguard
  \item
    peer discovery - for each pair of peers, one needs to have an
    endpoint that can be reached by the other peer
  \end{itemize}
\item
  Resources

  \begin{itemize}
  \tightlist
  \item
    \url{https://www.jordanwhited.com/posts/wireguard-endpoint-discovery-nat-traversal/}
  \end{itemize}
\end{itemize}

\hypertarget{noise-protocol-framework}{%
\section{Noise Protocol Framework}\label{noise-protocol-framework}}

\begin{itemize}
\tightlist
\item
  Framework for building protocols
\item
  Spec - \url{http://www.noiseprotocol.org/noise.pdf}
\item
  Suite of channel establishment protocols
\item
  Similar to TLS
\item
  Based on Elliptic-curve Diffie--Hellman (ECDH) Handshakes
\item
  Used in WhatsApp, Signal, Wireguard
\item
  Resources:

  \begin{itemize}
  \tightlist
  \item
    Crypto Layers

    \begin{itemize}
    \tightlist
    \item
      Low-level primitives - AES, ChaCha20.
    \item
      Usefully combined primitives - AES-OCB, NaCl secretbox\ldots{}
    \item
      High-level protocols - TLS, Noise.
    \item
      (sometimes) crypto spoken over the encrypted protocols, often for
      E2E crypto e.g.~GPG over SMTPS, or CloudFlare blinded CAPTCHA
      tokens over HTTPS
    \end{itemize}
  \item
    \url{https://noiseexplorer.com/}
  \item
    \url{https://duo.com/labs/tech-notes/noise-protocol-framework-intro}
  \item
    \url{https://www.youtube.com/watch?v=ceGTgqypwnQ}
  \end{itemize}
\end{itemize}

\hypertarget{tailscale}{%
\section{Tailscale}\label{tailscale}}

\begin{itemize}
\tightlist
\item
  Mesh VPN
\item
  Built on top of {[}{[}0203-wireguard\textbar Wireguard{]}{]}
\item
\item
  Coordination service

  \begin{itemize}
  \tightlist
  \item
    Closed source
  \item
    Facilitates
    \href{notes/0202-nat\#Session\%20Traversal\%20Utilities\%20for\%20NAT\%20(STUN)}{STUN}/\href{notes/0202-nat}{TURN}
    for peer discovery
  \item
    Distributes wireguard public keys
  \end{itemize}
\item
  Client

  \begin{itemize}
  \tightlist
  \item
    Open source
  \item
    Interacts with the Coordination service
  \item
    Configures Wireguard
  \end{itemize}
\end{itemize}

\hypertarget{nebula}{%
\section{Nebula}\label{nebula}}

\begin{itemize}
\tightlist
\item
  Mesh VPN
\item
  Similar to Tailscale
\item
  Does not use Wireguard
\item
  Built using the {[}{[}0204-noise\textbar Noise Protocol
  Framework{]}{]} (used in Wireguard)
\item
  Uses a Certificate Authority that needs to sign each Peer's
  certificate

  \begin{itemize}
  \tightlist
  \item
    Certificates contain

    \begin{itemize}
    \tightlist
    \item
      Peer's Virtual IP address
    \item
      Peer's public key
    \end{itemize}
  \end{itemize}
\end{itemize}

\hypertarget{openziti}{%
\section{OpenZiti}\label{openziti}}

\begin{itemize}
\tightlist
\item
  Network Overlay
\item
  No STUN
\item
  Works by relaying traffic through intermediaries
\item
  Focused on Services
\item
  Allows embedding into apps via an SDK
\end{itemize}

\hypertarget{didcomm-v2}{%
\section{DIDComm v2}\label{didcomm-v2}}

\begin{itemize}
\item
  Protocol for communicating that uses DID for identity management
\item
  Spec: \url{https://identity.foundation/didcomm-messaging/spec/}
\item
  No sessions

  \begin{itemize}
  \tightlist
  \item
    messages are always encrypted with the public keys of the peers
  \end{itemize}
\item
  Routing happens via mediators
\item
  \begin{longtable}[]{@{}l@{}}
  \toprule()
  sources: \\
  \midrule()
  \endhead
  -
  \url{https://newsletter.impervious.ai/decentralized-identifiers-implications-for-your-data-payments-and-communications-2/} \\
  \bottomrule()
  \end{longtable}
\end{itemize}

\hypertarget{ngrok}{%
\section{ngrok}\label{ngrok}}

\begin{itemize}
\tightlist
\item
  Paid service for creating public URLs for local services
\item
\end{itemize}

\hypertarget{bittorrent}{%
\section{BitTorrent}\label{bittorrent}}

\begin{itemize}
\tightlist
\item
  No concept of identity based communication or peer discovery
\end{itemize}

\hypertarget{ethereums-p2p-protocol}{%
\section{Ethereum's P2P Protocol}\label{ethereums-p2p-protocol}}

\hypertarget{ipfs}{%
\section{IPFS}\label{ipfs}}

\begin{itemize}
\tightlist
\item
  Content Addressable Storage
\item
  Network is for discovering data, not for executing computations
\item
\end{itemize}

\hypertarget{tor}{%
\section{TOR}\label{tor}}

\hypertarget{friend2friend-networks}{%
\section{Friend2Friend Networks}\label{friend2friend-networks}}

\begin{itemize}
\tightlist
\item
  Peer to Peer Dark Web network
\item
  Network that directly connects the machines of people who know each
  other
\item
  Services and files accessible only by other people on the network
\item
\end{itemize}

\hypertarget{brainstorm-for-custom-solutions}{%
\chapter{Brainstorm for custom
solutions}\label{brainstorm-for-custom-solutions}}

\begin{itemize}
\tightlist
\item
  Initial state

  \begin{itemize}
  \tightlist
  \item
    Here's my identity, here are the identities of the other parties
  \end{itemize}
\item
  Desired result

  \begin{itemize}
  \tightlist
  \item
    Executed MPC
  \end{itemize}
\end{itemize}

With tailscale we'd need to

\begin{itemize}
\tightlist
\item
  Each party:

  \begin{itemize}
  \tightlist
  \item
    registers a Tailscale account
  \item
    Downloads and installs tailscale on the machine they want to run the
    MPC on
  \item
    Runs tailscale on their machine and logs into their account in order
    to link it to their own Tailnet
  \item
    Shares their Tailscale machine with the Tailnets of each of the
    other parties
  \item
    Downloads the demo they want to run
  \item
    Form the flags for running the chosen demo

    \begin{itemize}
    \tightlist
    \item
      add -P \$HOST:\$PORT for each party using their Tailscale
      hostname/virtual IP
    \end{itemize}
  \item
    Run the demo
  \end{itemize}
\end{itemize}

\begin{center}\rule{0.5\linewidth}{0.5pt}\end{center}

\begin{itemize}
\tightlist
\item
  we don't need to use the same route for the communication Party A →
  Party B and Party B → Party A
\item
  we can have something like an asynchronous STUN
\item
  Party A sends a QR code/public URL/json object/DID document containing

  \begin{itemize}
  \tightlist
  \item
    Party A's public key
  \item
    Party A's Mediator URL
  \end{itemize}
\item
  The mediator is a STUN/TURN/DERP server

  \begin{itemize}
  \tightlist
  \item
    other parties can either use it as a STUN server to find out how to
    access the hole in party A's NAT punched by the mediator
  \item
    or use it as a relayer so that Party B can send encrypted packets to
    party A via the mediator
  \end{itemize}
\item
  Instead of a QR code, the information could be stored on a public
  ledger and could be resolved via DIDs
\end{itemize}

\begin{center}\rule{0.5\linewidth}{0.5pt}\end{center}

\begin{itemize}
\tightlist
\item
  There is a generic MPC wrapper program that deals with supporting
  tasks like generating identities and pulling MPC demos
\item
  One party creates a ``lobby'' for an MPC session in their program and
  get a session id/public URL/QR code that can be shared with the other
  parties
\item
\end{itemize}

\begin{center}\rule{0.5\linewidth}{0.5pt}\end{center}

\begin{itemize}
\tightlist
\item
  Public Website is visited by everyone
\item
  They prove their identities using a SSI wallet or OIDC
\item
  They get a QR code that serves as an invitation

  \begin{itemize}
  \tightlist
  \item
    contains their STUN based endpoint and identity
  \end{itemize}
\item
  Somehow everyone needs to scan each other's qr codes
\end{itemize}

\begin{center}\rule{0.5\linewidth}{0.5pt}\end{center}

\begin{center}\rule{0.5\linewidth}{0.5pt}\end{center}

\begin{center}\rule{0.5\linewidth}{0.5pt}\end{center}

\begin{itemize}
\tightlist
\item
  One party creates a ``lobby'' for an MPC session by visiting a public
  website

  \begin{itemize}
  \tightlist
  \item
    They provide their identity via OIDC/SSI wallet
  \end{itemize}
\item
  They get a public link/QR code that can be shared with the other
  parties
\item
  The parties visit the URL and also provide their identities
\item
  The parties obtain the MPC program they want to run

  \begin{itemize}
  \tightlist
  \item
    MPC program distribution could be done separately via cloning the
    github repo?
  \item
    They could choose a DEMO and download it from the website?
  \item
    There could also be a program running on the host machines that
    deals with the source code distribution. Similar to downloading
    custom maps for warcraft 3 or dota 2?
  \item
    They could specify the source code when creating the MPC session in
    the website?
  \item
    If the demo is not symmetrical where different parties have
    different roles and need to execute different programs, the roles
    could be assigned by the host or the people could choose their
    preferred role themselves?
  \end{itemize}
\item
  The parties download a configuration file that contains information on
  how to connect to the other parties
\item
  They run the demos with the downloaded config file
\item
  A temporary Wireguard mesh VPN is created between all parties
\end{itemize}

\hypertarget{latex}{%
\chapter{Latex}\label{latex}}

\begin{itemize}
\tightlist
\item
  use \textbackslash input instead of \textbackslash include to avoid
  issues with missing directories and .aux files
\end{itemize}
