\begin{frame}
\tableofcontents

\printnoidxglossary[type=\acronymtype,title=List of Abbreviations]
\listoffigures

\mainmatter
\end{frame}

\begin{frame}{Introduction}
\protect\hypertarget{introduction}{}
\end{frame}

\begin{frame}{Testing methodology}
\protect\hypertarget{testing-methodology}{}
During the preparation phase of the project we developed the \gls{e3}
framework which simplifies and automates the process of deploying
machines in different geographical regions, connecting them via an
overlay network and executing multiparty computations between them,
where each machine represents a different party. During the thesis
assignment we will look at a number of solutions for ad hoc MPC sessions
and compare them in terms of performance, security and usability.

\begin{block}{Performance}
\protect\hypertarget{performance}{}
To summarize, \gls{e3} is a set of scripts that use a number of
automation tools:

\begin{itemize}
\tightlist
\item
  Terraform - declarative provisioning
\item
  NixOS - declarative Linux distribution
\item
  Colmena - declarative deployment for NixOS
\item
  PSSH - parallel execution of remote scripts over ssh
\item
  DigitalOcean - a cloud provider
\end{itemize}

It allows us to quickly provision cloud virtual machines in multiple
regions and reproducibly deploy all necessary software for running a
multiparty computation over a chosen network overlay solution. The
source code of \gls{e3} can be found on
\href{https://github.com/e-nikolov/mpyc}{GitHub}

Each solution will be deployed using the \gls{e3} framework and the
performance will be quantitatively measured in terms of the time it
takes to execute a number of MPyC demos. The selected demos have
different complexities in terms of communication rounds and message
sizes which will allow us to observe their impact on the overall
performance.

\begin{enumerate}
\tightlist
\item
  Secret Santa - high round complexity with small messages
\item
  Convolutional Neural Network (CNN) MNIST classifier - low round
  complexity with large messages
\end{enumerate}

The demos will be configured at three different input size levels

\begin{itemize}
\tightlist
\item
  Low,
\item
  Medium
\item
  High
\end{itemize}

Furthermore, the demos will be executed in several networking scenarios:

\begin{enumerate}
\tightlist
\item
  1-10 parties in the same geographic region
\item
  1-10 parties evenly distributed across two nearby regions
\item
  1-10 parties evenly distributed across two distant regions
\item
  1-10 parties distributed across multiple distant regions
\end{enumerate}
\end{block}

\begin{block}{Security}
\protect\hypertarget{security}{}
We will analyze aspects such as

\begin{itemize}
\tightlist
\item
  key distribution
\item
  trust model - are there any trusted third parties and what would be
  the consequences if they are corrupted or breached
\item
  traffic encryption
\item
  identity strength
\end{itemize}
\end{block}

\begin{block}{Usability}
\protect\hypertarget{usability}{}
For each solution we will describe the steps that the parties need to
perform in order to execute a joint multiparty computation. Those steps
will be analyzed in terms of:

\begin{itemize}
\tightlist
\item
  Complexity - how much technical expertise is expected from the parties
  in order to be able to execute the steps
\item
  Initial effort - how much effort is each party expected to put in
  preparing for their first joint computation
\item
  Repeated effort - after the initial setup, how much effort is required
  to perform another computation

  \begin{itemize}
  \tightlist
  \item
    with the same set of parties
  \item
    with another set of parties
  \end{itemize}
\item
  Finalization effort - how much effort is required to finalize the MPC
  session once it is complete and clean up any left-over artifacts or
  resources so that the machine of each party is in its original state
\end{itemize}
\end{block}
\end{frame}

\begin{frame}{Overview}
\protect\hypertarget{overview}{}
In this chapter we will provide a high level overview of the solutions
that will be analyzed in more detail in the following chapters.

\begin{block}{Overlay Networking}
\protect\hypertarget{overlay-networking}{}
The Open Systems Interconnection (OSI) model distinguishes 7 layers in
computer networks:

\begin{itemize}
\tightlist
\item
  7 - Application layer
\item
  6 - Presentation layer
\item
  5 - Session layer
\item
  4 - Transport layer
\item
  3 - Network layer
\item
  2 - Data link layer -
\item
  1 - Physical layer
\end{itemize}
\end{block}

\begin{block}{Virtual Private Networks (VPN)}
\protect\hypertarget{virtual-private-networks-vpn}{}
\end{block}
\end{frame}

\begin{frame}[fragile]{Wireguard}
\protect\hypertarget{wireguard}{}
Wireguard\autocite{donenfeldWireGuardNextGeneration2017} is a VPN
protocol built with the Noise Protocol Framework\autocite{noiseProtocol}
that focuses on configuration simplicity. It considers issues such as
peer discovery and key distribution as out of scope and a responsibility
of a higher level system that uses Wireguard as a building block. The
snippets below show a minimal set of configuration options that need to
be provided in order for two peers to be able to form secure tunnels
with each another.

\begin{Shaded}
\begin{Highlighting}[]
\CommentTok{\# peer1.conf}
\KeywordTok{[}\DataTypeTok{Interface}\KeywordTok{]}
\DataTypeTok{Address} \OperatorTok{=} \FloatTok{101.0}\KeywordTok{.}\DataTypeTok{0}\KeywordTok{.}\DataTypeTok{1}\ErrorTok{/}\DataTypeTok{32}
\DataTypeTok{ListenPort} \OperatorTok{=} \DecValTok{53063}
\DataTypeTok{PrivateKey} \OperatorTok{=} \DataTypeTok{ePTiXXhHjvAHdWUr8Bimk30n0gh3m241RAzsN0JZDW0}\OperatorTok{=}

\KeywordTok{[}\DataTypeTok{Peer}\KeywordTok{]}
\DataTypeTok{PublicKey} \OperatorTok{=} \DataTypeTok{BSn0ejd1Y3bKuD}\ErrorTok{+}\DataTypeTok{Xpg0ZZeOf}\ErrorTok{+}\DataTypeTok{Ies}\ErrorTok{/}\DataTypeTok{oql0NZxw}\ErrorTok{+}\DataTypeTok{SOmkc}\OperatorTok{=}
\DataTypeTok{AllowedIPs} \OperatorTok{=} \FloatTok{101.0}\KeywordTok{.}\DataTypeTok{0}\KeywordTok{.}\DataTypeTok{2}\ErrorTok{/}\DataTypeTok{32}
\DataTypeTok{Endpoint} \OperatorTok{=} \DataTypeTok{peer1}\KeywordTok{.}\DataTypeTok{example}\KeywordTok{.}\DataTypeTok{com}\ErrorTok{:}\DataTypeTok{38133}
\end{Highlighting}
\end{Shaded}

\begin{Shaded}
\begin{Highlighting}[]
\CommentTok{\# peer2.conf}
\KeywordTok{[}\DataTypeTok{Interface}\KeywordTok{]}
\DataTypeTok{Address} \OperatorTok{=} \FloatTok{101.0}\KeywordTok{.}\DataTypeTok{0}\KeywordTok{.}\DataTypeTok{2}\ErrorTok{/}\DataTypeTok{32}
\DataTypeTok{ListenPort} \OperatorTok{=} \DecValTok{38133}
\DataTypeTok{PrivateKey} \OperatorTok{=} \DataTypeTok{sN}\ErrorTok{/}\DataTypeTok{d6XUPEVPGSziVgCCOnOivDK}\ErrorTok{+}\DataTypeTok{qAoYC3nxnssQ5Rls}\OperatorTok{=}

\KeywordTok{[}\DataTypeTok{Peer}\KeywordTok{]}
\DataTypeTok{PublicKey} \OperatorTok{=} \DataTypeTok{e}\ErrorTok{/}\DataTypeTok{TxvPmrgcc1G4cSH2bHv5J0PRHXKjYxTFoU8r}\ErrorTok{+}\DataTypeTok{G93E}\OperatorTok{=}
\DataTypeTok{AllowedIPs} \OperatorTok{=} \FloatTok{101.0}\KeywordTok{.}\DataTypeTok{0}\KeywordTok{.}\DataTypeTok{1}\ErrorTok{/}\DataTypeTok{32}
\end{Highlighting}
\end{Shaded}

Each peer has a public/private key pair that is used for authentication
and encryption. The Address field specifies the virtual IP address that
the local network interface will use, while the AllowedIPs specifies
what virtual IP addresses are associated with a peer's public key. A
peer's Endpoint field specifies the URL at which it can be reached. Only
one of the peers must be configured with a reachable endpoint for the
other peer. In the above example once \texttt{peer1} initiates
communication with \texttt{peer2}, \texttt{peer2} will learn the current
endpoint of \texttt{peer1} and will be able to communicate back with it.

\begin{block}{Implementation}
\protect\hypertarget{implementation}{}
\end{block}
\end{frame}

\begin{frame}{Tailscale}
\protect\hypertarget{tailscale}{}
Tailscale is a VPN solution that configures a mesh of direct Wireguard
tunnels between the peers.

\begin{block}{Overview}
\protect\hypertarget{overview}{}
\end{block}

\begin{block}{Usability}
\protect\hypertarget{usability}{}
With tailscale each party needs to

\begin{itemize}
\tightlist
\item
  register a Tailscale account
\item
  Download and install tailscale on the machine they want to run a
  multiparty computation
\item
  Run tailscale on their machine and logs into their account in order to
  link it to their own Tailnet
\item
  Share their Tailscale machine with the Tailnets of each of the other
  parties
\item
  Download the demo they want to run
\item
  Form the flags for running the chosen demo

  \begin{itemize}
  \tightlist
  \item
    add -P \$HOST:\$PORT for each party using their Tailscale
    hostname/virtual IP
  \end{itemize}
\item
  Run the demo
\end{itemize}
\end{block}

\begin{block}{Security}
\protect\hypertarget{security}{}
\begin{block}{Trust model}
\protect\hypertarget{trust-model}{}
There is a centralized service that deals with the key distribution,
which needs to be trusted to provide the correct public keys for the
correct parties
\end{block}

\begin{block}{Identity}
\protect\hypertarget{identity}{}
Identity is based on third party identity providers such as Microsoft
and GitHub

\begin{itemize}
\tightlist
\item
  Magic DNS
\end{itemize}
\end{block}
\end{block}

\begin{block}{Performance}
\protect\hypertarget{performance}{}
\end{block}
\end{frame}
