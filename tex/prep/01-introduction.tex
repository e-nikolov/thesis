\hypertarget{introduction}{%
\chapter{Introduction}\label{introduction}}

This report will present the results of the preparation phase of the
master thesis project titled ``Secure Sessions for Ad Hoc Multiparty
Computation in MPyC''. The goal of this phase is to gain sufficient
insight into the topic, perform some preliminary tasks and propose a
plan with well defined goals for the implementation phase of the
project.

\hypertarget{background}{%
\section{Background}\label{background}}

\gls{mpc} is a set of techniques and protocols for computing a function
over the secret inputs of multiple parties without revealing their
values, but only the final result. A good overview can be found on
Wikipedia\autocite{wikiMPC}. Yao's Millionaires'
Problem\autocite{yaoProtocolsSecureComputations1982} is one famous
example in which a number of millionaires want to know who is richer
without revealing their net worths. Other practical
applications\autocite{laudApplicationsSecureMultiparty2015} include
electronic voting, auctions or even machine
learning\autocite{knottCrypTenSecureMultiParty2022} where one party's
private data can be used as an input for another party's private machine
learning model.

The general process is that each party uses a scheme like \gls{sss}
\autocite{shamirHowShareSecret1979} to split its secret input into
shares and sends one to each of the other parties. A protocol involving
multiple communication rounds and further re-shares of intermediate
secret results is used by the parties so that each of them can compute
the final result from the shares it has received.

A number of \gls{mpc} frameworks have been developed for various
programming languages and security models. As part of this project we
will focus our efforts on \textbf{MPyC}\autocite{mpycHome,mpycSource} -
an opensource \gls{mpc} Python framework developed primarily at TU
Eindhoven, but our results should be applicable to others as well.

To help us determine the types of solutions we need to consider, we can
group the potential users of the MPyC framework into three broad
categories: casual users, power users and enterprises.

We define \textbf{casual users} as people who are used to Windows or
Mac, prefer software installers to package managers, \gls{gui} programs
rather than \gls{cli} based ones and do not feel comfortable with
manually modifying their systems or using scripts.

We define \textbf{power users} as users who manage a number of personal
physical machines and may have some familiarity with Linux, terminals
and shell scripting. They are assumed to be able to execute the
necessary steps to setup a machine given a guide.

For our purposes we define \textbf{enterprises} as companies with
operations departments that manage their IT infrastructure and optimize
for scale. They usually have large numbers of Linux based servers which
are a combination of physical, virtual and container based. Those can be
deployed either in the cloud or on premise in an automated way using
\gls{iac} tools.

\hypertarget{problem-description}{%
\section{Problem description}\label{problem-description}}

MPyC supports \gls{tcp} connections from the \gls{ip} suite between the
\gls{mpc} participants but it does not currently provide a service
discovery mechanism. Before performing a joint computation, all parties
must know and be able to reach each other's \gls{tcp} endpoints - either
via a local \gls{ip} address on a \gls{lan}, or via a public \gls{ip}
address or the \gls{dns} on the internet. This is not likely to pose a
problem for most enterprise users, who are usually already exposing some
public services, e.g.~their website. However, most casual users who are
not \gls{it} experts typically do not own a domain name nor know how to
configure a publicly accessible server. Due to the limited supply of
addresses supported by \gls{ip}v4 and the slow adoption of \gls{ip}v6,
most \glspl{isp} do not allocate a public address for each machine in
the home networks of their residential customers. Usually only their
router has a temporary public address and it utilizes techniques such as
\gls{nat} in order to enable the other local machines to initiate remote
internet connections. However, connections to them cannot be initiated
from outside the \gls{lan} without manually configuring port forwarding
in the router to send the appropriate traffic to the intended machine.
This poses some challenges and limits the usability of MPyC in every day
scenarios due to the inherently \gls{p2p} nature of the involved
communications.

\hypertarget{research-questions}{%
\section{Research questions}\label{research-questions}}

Based on the problem description above, we formulate the following
central research question:

\emph{How can MPyC be extended to enable casual users, power users and
enterprises with limited prior knowledge of each other to discover each
other and perform a secure multiparty computation under diverse
networking conditions?}

We further identify the following sub-questions:

\begin{itemize}
\tightlist
\item
  \emph{What deployment strategies should be supported in order to
  accommodate the potential users of MPyC programs?}
\end{itemize}

Depending on the technical background of a user and their typical
computing usage, they might have different expectations for how to
execute their part of the joint multiparty computation, e.g.~enterprises
might expect support for automation tools, while casual users could
expect simplicity. There should be some safety (not necessarily
security) mechanism for detecting and preventing mistakes where the
users are accidentally executing incompatible MPyC programs/versions.

\begin{itemize}
\tightlist
\item
  \emph{What are the most suitable approaches for a party to obtain an
  identity and prove it to other parties for the purposes of MPC?}
\end{itemize}

A party's digital identity is a persistent mechanism that allows others
to provably track it across digital interactions with the party. An
identity can either be issued by a digital authority, e.g.~an
organization like google, or a country's government or it can be
self-issued. Depending on the method, an identity verification can
involve demonstrating a cryptographic proof of ownership of a public
key, or separate communication with the digital authority.

\begin{itemize}
\tightlist
\item
  \emph{What mechanisms can be used by the parties to initially get in
  contact and discover each other's identities?}
\end{itemize}

Different approaches should be considered based on the types of users
and their prior relationships. Some examples could be for companies to
publicly post their identities on their websites, end-users who know
each other could use social media or group chats and if they do not know
each other, they could use an (anonymous) matchmaking service.

\begin{itemize}
\tightlist
\item
  \emph{How can the parties establish communication channels with each
  other based on the chosen identity solutions under diverse networking
  conditions?}
\end{itemize}

As previously mentioned, some parties could be on a home network and not
have a public \gls{ip}, which may require considering approaches that
use a mediator.

\begin{itemize}
\tightlist
\item
  \emph{How can the parties communicate securely as part of the MPC
  execution? To what extent can the parties' privacy be preserved? How
  efficiently can this be achieved?}
\end{itemize}

In order for the execution of an \gls{mpc} protocol to be secure, it is
important for the parties to be able to cryptographically verify the
identity of a message's original sender and be certain that nobody other
than themselves can read it. Solutions that do not reveal physically
identifying information such as \gls{ip} addresses are also interesting
to consider. The performance overhead of the security mechanisms should
be evaluated.

\hypertarget{preparation-phase-scope}{%
\section{Preparation phase scope}\label{preparation-phase-scope}}

During the implementation phase, we will answer the posed research
questions of the project after evaluating various connectivity
approaches for MPyC. The scope of the preparation phase will cover a
technical survey to identify some of the tools that can be used and the
development of an \gls{e3} that will support the evaluation process in
the next phase. \gls{e3} must enable fearless experimentation with
different implementations of the connectivity layer. \gls{e3} will focus
on being reproducible by enterprise and power users while keeping it
representative of real world scenarios involving casual users as well.

Below, we formulate our requirements for \gls{e3} and group them in
terms of several important characteristics:

\begin{itemize}
\tightlist
\item
  Complexity

  \begin{itemize}
  \tightlist
  \item
    simple - given the limited time of the preparation phase, \gls{e3}
    must focus on simplicity, e.g.~the easiest to implement connectivity
    approach should be chosen
  \item
    extensible - \gls{e3} must allow for switching the building blocks
    during the next phase of the project, e.g.~it should be easy to
    experiment with different connectivity approaches in order to
    measure and compare their characteristics
  \end{itemize}
\item
  Source code

  \begin{itemize}
  \tightlist
  \item
    open-source - the source code of the resulting implementation of
    \gls{e3} must be available in a public repository, e.g.~on
    Github.com
  \item
    no plaintext secrets such as \gls{api} keys and passwords should be
    present in the public repository, but others should be able to
    easily provide their own secrets in order to use \gls{e3} in their
    own environment.
  \end{itemize}
\item
  Deployment

  \begin{itemize}
  \tightlist
  \item
    cross region - the machines should be provisioned in multiple
    geographical regions in order to be able to observe the effects of
    varying latency on the system
  \item
    cross platform - in a real world scenario the machines will be
    controlled by different parties that run various operating systems,
    hardware architectures and deployed using different tools,
    e.g.~Party A might be an enterprise that uses containers, while
    Party B is a power user running a few \glspl{vm} and Party C
    ****could be using an ARM-based raspberry pi
  \item
    automated - appropriate tools should be chosen to allow
    automatically deploying and destroying the runtime environment
    without manual intervention other than running a minimal set of
    commands
  \item
    reproducible - it should be easy for others to reproduce the test
    setup in their own environment
  \item
    disposable - \gls{e3} should be easy to destroy and quickly recreate
    from scratch at any time; as in the famous DevOps
    analogy\autocite{biasHistoryPetsVs2016}, it should be based on
    machines that are like cattle rather than pets
  \end{itemize}
\item
  Connectivity

  \begin{itemize}
  \tightlist
  \item
    identity - it must be possible for the machines to communicate based
    on a long-lived identity rather than a potentially temporary
    \gls{ip} address.
  \item
    secure - a message sent by a party must be readable only by its
    intended targets.
  \item
    authenticated - a party must be able to determine which party a
    message was sent by
  \item
    private - no more information than strictly necessary should be
    revealed about a party. Depending on the method of communication, it
    may be necessary to choose a tradeoff or introduce a tuning
    parameter between performance and privacy.
  \end{itemize}
\end{itemize}
