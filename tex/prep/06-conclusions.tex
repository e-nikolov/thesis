\hypertarget{conclusions}{%
\chapter{Conclusions}\label{conclusions}}

In this report we presented the results of the preparation phase for the
master thesis assignment ``Secure Sessions for Ad Hoc Multiparty
Computation in MPyC''. We developed an \acrfull{e3} for the purpose of
creating ad hoc networks of host machines that perform \acrfullpl{mpc}
in hybrid scenarios involving both cloud and physical machines. \gls{e3}
makes extensive use of declarative \gls{iac} tools in order to achieve
highly reproducible deployments in an automated way. We provided a
reference implementation that makes use of the Tailscale mesh VPN that
creates a network of RaspberryPis and cloud \glspl{vm} on DigitalOcean.
The cloud provisioning is defined declaratively using Terraform and
allows to define a set of host machines across the regions supported by
DigitalOcean (e.g.~Amsterdam, New York City, etc) and automatically add
them to a shared Tailscale network. The machines run NixOS - a
declarative and highly reproducible Linux distribution while Colmena is
used to declaratively manage the software installed on them via
\gls{ssh}. The tools \texttt{prsync} and \texttt{pssh} are used to run
MPyC demos in parallel on the deployed hosts.

\hypertarget{implementation-phase-planning}{%
\subsubsection{Implementation phase
planning}\label{implementation-phase-planning}}

During the next phase of the thesis assignment, we plan to implement
various connectivity approaches for \gls{e3}'s host machines and analyse
their suitability for MPyC.

The following is a list of high level tasks that we plan to carry out as
part of the implementation phase:

\begin{itemize}
\tightlist
\item
  replace the proprietary Tailscale coordination service from our
  reference implementation of \gls{e3} with the open-source self-hosted
  alternative Headscale\autocite{fontJuanfontHeadscale2022}
\item
  develop a network overlay for \gls{e3} based on the Nebula mesh VPN.
  Nebula only provides a way to manually perform the initial setup, so
  our implementation should add a way to automatically:

  \begin{itemize}
  \tightlist
  \item
    allocate virtual IP addresses for the hosts
  \item
    generate identity certificates using the Nebula \gls{ca}
  \item
    distribute the certificates among the hosts
  \end{itemize}
\item
  develop network overlays for \gls{e3} that incorporate parts of the
  mesh VPN implementations but with alternative identity management
  approaches:

  \begin{itemize}
  \tightlist
  \item
    using a \gls{ca} that is managed jointly using MPC
  \item
    using a form of \gls{ssi} such as \glspl{did}
  \end{itemize}
\item
  implement a network overlay for \gls{e3} based on DIDComm
\item
  explore options for enhancing the DIDComm implementation to:

  \begin{itemize}
  \tightlist
  \item
    support sessions - the DIDComm protocol is currently stateless and
    uses a new asymmetric key for each message, which negatively impacts
    performance
  \item
    employ \gls{nat} traversal techniques similar to mesh \glspl{vpn}
  \end{itemize}
\item
  implement a privacy mechanism for \gls{e3} based on \gls{tor} in order
  to prevent leaking sensitive information like which parties are
  communicating with each other and their IP addresses
\item
  investigate if we can apply ideas from the \gls{p2p} implementations
  in other software like the
  Ethereum{[}\textcite{ethereumDocs}{]}\autocite{ethereumYellowPaper}
  blockchain and the \gls{ipfs} \autocite{ipfsDocs}
\item
  analyse and compare all of the above implementations in terms of:

  \begin{itemize}
  \tightlist
  \item
    security
  \item
    performance
  \item
    ease of use
  \item
    privacy
  \end{itemize}
\item
  compare \gls{e3} to other work related to deploying MPC such as the
  Carbyne stack\autocite{robertboschgmbhCarbyneStack2022}
\end{itemize}
